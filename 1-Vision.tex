\documentclass{article}
\usepackage{graphicx} % Required for inserting images
\usepackage{paralist} % Required for compact lists
\graphicspath{ {./images/} }

\title{Coordimate\\Vision Document}
\date{December 2023}
\author{Ali Bassam, Vlada Prikhodchenko, Valentin Vikhorev}

\begin{document}

\maketitle

\begin{center}
\includegraphics[width=8cm]{coordimate-logo}
\end{center}

\section{Introduction}

\textit{In this section provide introductory details about the software
(application, system, etc.). This can include where the inspiration for the
initial idea originated.}

Coordimate is a scheduling app that aims to ease the finalisation of meetings
according to availability marked by users on their personal in-app schedules.

\section{Product / Solution Overview}

\textit{In this section provide an overview of what solutions are provided by
the software. Who is the target audience? How will the software help them? What
are the expected improvements (time, efficiency, ease of access, etc.) to be
expected by the target audience?}

While other applications of this kind (Lettuce meet) do exist, they often have
to face the problem of the users indicating which times they prefer to meet,
rather than which times they’re actually free. To solve this, Coordimate
prompts users to create a schedule indicating their busy hours during the week,
instead of their free ones, allowing the user to create and share a link of
their schedule in groups they are a part of, where the app automatically
highlights the common availability of the selected participants.

\section{Business Needs / Requirements}

\textit{Provide here a brief overview of the user requirements/needs that the
software will solve. The detailed list of requirements will be outlined in the
requirements document.}

\begin{compactenum}% 
    \item Share a user's busy-time schedule
    \item Create and manage project groups and group calendars
    \item Set up meetings with an agenda, and notify group members
    \item Chat with group members, hold polls, and store meeting notes
\end{compactenum}

\section{Major Features}

\textit{Many applications or systems have multiple parts and features. Here
describe briefly each of the major features that will provide the user with
functionality.}

\begin{itemize}
    \item{Create a personal schedule, selecting busy hours according to predefined
        hours of days in a week.}
    \item{Create and share a link of the personal schedule to share with other
        users.}
    \item{Create, join, and share groups, for the purpose of indicating common
        available time slots according to all (or a selection of) the group
        members’ schedules.}
    \item{Prompt, set-up and confirm meetings according to a selected constraint
        (earliest date possible, after 6pm, not on weekends, etc…)}
    \item{Integration with meeting platforms (Google Meet, Webex, etc) through API
        to create scheduled meetings.}
    \item{Having meeting link available in the app, and only accessible/visible to
        members of the group that voted for the meeting (access can be modified
        according to group admins)}
    \item{Suggestions for meeting times based on historical data, user preferences,
        and re-filling schedules.}
    \item{Synchronise external calendars (Google Calendar, Outlook, etc.)}
    \item{Automated reminders for upcoming meetings, changes in schedules, or
        pending meeting confirmations.}
    \item{Integrate location-based services - the app suggests cafes for your group
        to meet up offline (approximately same distance from all group members).}
    \item{In-app messaging system to facilitate communication.}
    \item{Polls within the app regarding meetings preferences, creating generic polls.}
    \item{Provide a summary of the meeting based on the meeting transcript.}
    \item{Integrate "link-placeholder" feature that allows users to have links
        pre-saved in each meeting slot.}
    \item{Offline Mode (change and edit schedule without internet connection).}
    \item{Meeting agenda: set up a meeting agenda with bullet points for tasks.
        Keep track of timestamps when meeting points are crossed out. Each meeting
        has optional fields for the title and the agenda. The agenda is organised
        as a multi-level bulleted list.}
    \item{(bonus) “Random coffee”: if a person wants to go for a coffee break,
        they can place special time slots. If another person from one of the teams
        of the user also has slots for coffee at the same time, they get
        matched and receive notifications. They can accept or decline the
        invitations. Users can add people to “blacklist” if they don’t want to
        be matched with them.}
\end{itemize}

\section{Scope and Limitations}

\textit{It is good to define a scope for software development to avoid excess
feature drift and detect when potential redesign maybe needed due to an
increasing scope. Also, list the limitations of the proposed architecture in
order to identify some of the shortfalls of the proposed system.}

The proposed software system consists a of cross-platform mobile application, 
and a backend API service. The system is limited in scope to scheduling
meetings, and providing utilities for team collaboration during meetings.

The system is not a meeting platform, nor is it a full calendar suite.

\end{document}

