\documentclass{article}
\usepackage{graphicx} % Required for inserting images

\title{Coordimate\\Test Plan}
\date{December 2023}
\author{Ali Bassam, Vlada Prikhodchenko, Valentin Vikhorev}

\begin{document}

\maketitle

\section{Testing Strategy}

\subsection{Overall strategy}

\textit{Describe the overall testing strategy. In this
template only a few general all purpose categories are listed, feel free to
expand and add to these for your specific development needs.}

\subsubsection{Unit Testing}

\textit{Most development, especially in an agile setting,
uses unit testing. Here describe the unit testing that will be performed. This
can involve list and describing the unit testing framework(s) that will be used
and for which part of the application development the framework will be used.
If there will be any general strategies being employed or specific guidelines
being followed, these should be listed here as well.}

\subsubsection{Integration Testing}

\textit{Most applications will have multiple
parts that are connected by some sort of adapter or connector. In this section
describe how the testing of part integration will take place (manually, unit
tests, etc.).}

\subsubsection{System Testing}

\textit{Applications will run on one or multiple
systems. Checks to ensure that they do not break at the system level must be
conducted. This step does not have to be conducted often but at least once
before each release. Describe here the method for testing that there are no
incompatibilities between the system and application.}

\subsubsection{Regression Testing}

\textit{During continuing development as new
features and functionality are added periodic testing must be conducted in an
attempt to discover and detect new bugs or incompatibilities that arisen since
the last release/commit. Here describe the strategy for testing the code to
find these bugs. This is often done with unit tests or similar and performed
before each commit and after each update.}

\subsection{Test Selection}

\textit{Here describe the strategy for selecting and identifying which areas of
the application need to be tested. Where will textbf{white-box} testing be used
and where will textbf{black-box} testing be used.}

\subsection{Bug Tracking}

\textit{Here describe the strategy/tools that will be used
to handle and record bugs and their fixes. This can be a combination of things
that will lead to a more robust system.}

\subsection{Technology}

\textit{Here list/describe any additional additional relevant
information about the tools and systems that will be used to conduct the
testing of the application and its dependencies.}


\section{Test Cases}

\textit{In this section describe the test cases for the
application and list which tool they are being conducted using. No subsections
are given because they are to be filled in by the user. This section will be
constantly expanding as development progresses. The purpose is to be a
reference for both application developers and test developers. Expand as team
sees fit.}

\end{document}

